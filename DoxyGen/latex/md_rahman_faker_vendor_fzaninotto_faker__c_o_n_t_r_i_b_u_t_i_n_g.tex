If you\textquotesingle{}ve written a new formatter, adapted Faker to a new locale, or fixed a bug, your contribution is welcome!

Before proposing a pull request, check the following\+:


\begin{DoxyItemize}
\item Your code should follow the \href{https://github.com/php-fig/fig-standards/blob/master/accepted/PSR-2-coding-style-guide.md}{\texttt{ PSR-\/2 coding standard}} (and use \href{https://github.com/fabpot/PHP-CS-Fixer}{\texttt{ php-\/cs-\/fixer}} to fix inconsistencies).
\item Unit tests should still pass after your patch
\item As much as possible, add unit tests for your code
\item If you add new providers (or new locales) and that they embed a lot of data for random generation (e.\+g. first names in a new language), please add a link to the reference you used for this list (example \href{https://github.com/fzaninotto/Faker/blob/master/src/Faker/Provider/ru_RU/Person.php\#L13}{\texttt{ in the ru\+\_\+\+RU locale}}). This will ease future updates of the list and debates about the most relevant data for this provider.
\item If you add long list of random data, please split the list into several lines. This makes diffs easier to read, and facilitates core review.
\item If you add new formatters, please include documentation for it in the README. Don\textquotesingle{}t forget to add a line about new formatters in the {\ttfamily @property} or {\ttfamily @method} php\+Doc entries in \href{https://github.com/fzaninotto/Faker/blob/master/src/Faker/Generator.php\#L6-L118}{\texttt{ Generator.\+php}} to help IDEs auto-\/complete your formatters.
\item If your new formatters are specific to a certain locale, document them in the \href{https://github.com/fzaninotto/Faker\#language-specific-formatters}{\texttt{ Language-\/specific formatters}} list instead.
\item Avoid changing existing sets of data. Some developers use Faker with seeding for unit tests ; changing the data makes their tests fail.
\item Speed is important in all Faker usages. Make sure your code is optimized to generate thousands of fake items in no time, without consuming too much memory or CPU.
\item If you commit a new feature, be prepared to help maintaining it. Watch the project on Git\+Hub, and please comment on issues or PRs regarding the feature you contributed.
\end{DoxyItemize}

Once your code is merged, it is available for free to everybody under the MIT License. Publishing your Pull Request on the Faker Git\+Hub repository means that you agree with this license for your contribution.

Thank you for your contribution! Faker wouldn\textquotesingle{}t be so great without you. 